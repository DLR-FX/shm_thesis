% !TeX root = main.tex
% Themenstellung / Topic short description
\pagestyle{empty}
~
\newpage
\begin{tikzpicture}[remember picture,overlay]
	\node[opacity=0.4, anchor=north east,inner sep=0pt]%
	at ($(current page.north west)+(20.5cm,-0.5cm)$)
	{\includegraphics[width=7.67cm]{DIGITAL_WZL-Logo-Blau.jpg}};
\end{tikzpicture}
\parindent=0em

Aachen, \verbosedate\today\\
%V. Name - Tel. 0241-80 xxxxx\\[2cm]

{\huge \bf Master Thesis}\\[1cm] %% EN

% {\huge \bf Masterarbeit}\\[1cm] %% DE

\begin{tabbing}
%% EN
for B.Sc. \quad \= Colin Klein\\
\> Matriculation number: 368642\\

%% DE 
%für Frau/Herrn Cand.-Ing. \quad \= Erika Mustermann\\
%\> Matrikelnummer: 081511\\
	
\end{tabbing}

\begin{minipage}[t]{0.1\textwidth}
Topic: %% EN
%Thema: %% DE
\quad
\end{minipage}
\begin{minipage}[t]{0.9\textwidth}
	FAIR Sensor Health Monitoring of Flight Test Data.\\
\end{minipage}

The DLR's ISTAR research aircraft is equipped with extensive permanent sensor instrumentation for the scientific investigation of aerophysical phenomena. The measurement data as well as a large part of the parameters on the aircraft's own data bus are continuously recorded by an additional Data Acquisitioning System (DAQ). To evaluate measurement data in order to gain knowledge, the form of data storage and the linkage of fault detections is vital to avoid erroneous conclusions.
Honoring FAIR (Findable, Accessible, Interoperable, Reusable) principles, the detected Faults and Anomalies shall aptly be linked to the measurement data.
In this work, a python application is developed to solve this problem. Faults are detected, sensibly linked to the measurement data and then visualized to the user.
To solve this task, data modelling techniques developed at the WZL are employed. In addition, data is checked for completeness, plausibility and correctness by using statistical methods as well as approaches from the field of Fault Mode and Effect Analysis (FMEA).
The performance and trueness of the application-toolchain is then tested against known errors and validated on a dataset of ISTAR Flight Test data.
\\


\vspace{4cm}

Prof. Dr.-Ing. Robert Schmitt

