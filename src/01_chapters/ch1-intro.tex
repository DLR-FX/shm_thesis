% !TeX root = main.tex

%todo: check ob kommentar von christina erledigt ist?
%Hier könntest du auch hervorheben, dass der Wert einer solchen Datanbank gering ist, solang nicht sichergestellt werden kann, dass alle darauf liegenden Daten valide sind. Ein bloßes herausfiltern korrupter Daten würde unserer Philosophie wiedersprechen. Unser Anspruch ist die Daten so bereitzustellen wie sie gemessen wurden, dabei aber Messfehler und Messunsicherheiten sichtbar zu machen so dass der Nutzer sich auf die Qualität unserer Messdaten verlassen kann.
\chapter{Introduction \label{ch1-intro}}
%1. state the general topic and give some background
%a. It is about checking Flight Test Data. Within the FX-FTI department data is generated by aircraft. And right now it is also getting uploaded to the skystash platform to provide easier access to users instead of a long and tedious bureaucratic chain
Experimental Research Data is essential to confirm or nullify scientific theories. Motivated by the drive for sustainability the idea of Research Data Reuse culminated in the 2016 FAIR Guiding Principles \cite{wilkinson_fair_2016}. These principles stand for Findable, Accessible, Interoperable and Reusable data signifying attributes that enable the sharing and colaboration of data. These principles then enable data reuse to enable Open Data Access and Big Data within the scientific field \cite{hodson_fair_2018}. In this work research data of the ISTAR (In-flight Systems \& Technology Airborne Research) aircraft of the DLR (German Aerospace Center) will be examined to detect faults present in the data. This improves reusability and will notify future users of any known anomalies within the data in order to gain insights into the data and increase the data quality. In the context of the data producer and operator of the measuring system it is also vital to increase the fault detection rate and decrease detection time since sensor malfunctions are at best detected late. Commonly, a sensor- or system-failure may not be detected during a flight experiment but weeks after. The ensuing rescheduling within the aircraft's full timetable makes it even more economically important to implement a system to catch sensor faults and failures to avoid the need for an experiment to be repeated. In a worst case scenario a whole experiment would need to be replicated with the special aircraft configuration of custom sensor equipment needing to be recustomized and refitted which is a time- and funding intensive task.
A Sensor Health Monitoring software would enable detecting errors imminently after the sensor data is extracted from the aircraft. Due to the lower reaction time, errors may be detected and diagnosed on the same day, allowing for insights into the data that are currently inaccessible.

In previous works, a database has been created and the Flight Test Data has been provisioned. This database and the infrastructure used in this work is the german aerospace centre's (DLR) digital twin project and the emerging skystash platform \cite{meyer_development_2020, arts_digital_2022}. The skystash will be used to centralize data operations and enable distribution digitizing the data and digitalizing the convoluted internal processes. Next to data handling metadata handling is of great importance within this work's context to cleanly interface the different components that will be developed and to avoid high coupling of singular components. Fortunately, the Internet of Things (IoT) as well as the Internet of Production (IoP)\cite{pennekamp_towards_2019} and Digital Twin contexts provide a vast and rapidly expanding field of such methods. Notable works for metadata are the Asset Administration Shell (AAS) \cite{bader_details_2020} and the SensOr Interfacing Language (SOIL) \cite{behrens_domain-specific_2021, bodenbenner_model-driven_2022} in IoP contexts that currently developed at the Chair of Production Metrology and Quality Management (MQ) at the Laboratory for Machine Tools and Production Engineering (WZL) of RWTH Aachen University (WZL-MQ).

Algorithms to detect faults can now be embedded into this landscape of data and metadata. A majority of these algorithms are developed for system control in the field of control theory and Failure Mode and Effects Analysis (FMEA) and thus are real-time applications that aim to maintain system control \cite{isermann_fault-diagnosis_2006}. Since the goal of this work is to quantify data quality these algorithms will have to be modified and can't be used out of the box which will form an essential part of this work. Furthermore, methods from the field of statistics and data statistics are examined for employability within this work \cite{handl_multivariate_2017}.


%-implement a FMEA
%-embed the FMEA into the FAIR data lifecycle using the skystash


%3. define the terms and scope of the topic
%a. Software needs to be developed to check data
%b. Focus on integration and implementation of simple FMEA
Within the scope of this work a prototype for a Sensor Health Monitoring (SHM) application is developed. First, the data gets examined to generate sample malfunctions upon which the FMEA algorithms can be trained. The FMEA algorithms are then developed based on these test cases. The developed algorithms then get embedded into a data pipeline process using FAIR contexts. Within this pipeline, data gets generated, enriched, checked, reported on and finally visualized. Since the software and FAIR development will take up a significant part of this work a FMEA method will be developed for a smaller subset of the actual sensors in order to deliver a holistic and heuristic approach that is scalable and adaptable.

%5. evaluate the current situation (advantages/ disadvantages) and identify the gap
%a. Disadvantage: no good error detection, no FAIR
%b. Advantage: no negative reliability, hence illusion of reliable data (perhaps necessating investments of funds and time into sensors that are found to be unreliable)

%6. identify the importance of the proposed research
%○ Costly mistakes/faults
%○ Better overview
%○ Quicker reaction time
%○ Fulfillment FAIR (quick overview)
%○ Safety critical errors may be reported


%11. outline the methodology
%i. Turbine model will be used for exploration and implementation of methods
%ii. FAIR
%iii. High coherency, low coupling

%\paragraph{Motivation}

% motivation?
% 1. actually detect data errors
% 2. promptly detect errors. manual diagnosis steps --> reduce downtime
% 3. increase quality of flight test data by marking errors to detect them



%Q: Why is this important?
%It is then important to reduce manual overview of sensors. Currently, much manual postprocessing is needed to get an overview over the large datasets. Large datasets however are difficult to work with due to being too large for a single computer. Necessitating a server architecture for work with this size of data. Another benefit of this work may be to reduce downtime of the Aircraft due to sensors. A diagnostic tool for detecting sensor behavior can facilitate the processes and allow a quicker follow up to detect sensor errors since currently system information is distributed and not clearly perceivable.

%Safety critical errors may be detected earlier since the proposed routine has potential to go beyond the scope of commercially tested software that is generally employed within aircraft and generally rudimentary resulting from the desire to employ very stable software. Once a baseline routine is in place it is then possible to build a foundation for a reliability index for sensors. This is vital for big data operations within digitalization and the skystash project. This should be able to implemented by plug and play into this works results. Allowing for quick implementation of custom data quality algorithms.

%Another motivation for this work is the fulfillment of the FAIR (Findable, Accessible, Interoperable, Reusable) principles. FAIR principles are the new standard within the DLR for research data management. The use within operations however are far from the fulfillment. Documents containing descriptions of sensors and setups are distributed among multiple employees obstructing any efforts to comprehend the already complex aircraft system and its inherent generation of data.

%The proposed work is then nothing short of essential, building a tool for detection and avoidance of sensor failures to mitigate risks associated with experimental setups since aircraft hours are expensive and the amount of sensors is vast.


%Research Introduction 7. state the research problem/ questions a. Implementation of a FAIR SHM 8. state the research aims and/or research objectives a. Finding of common data quality descriptors and definitions b. Find a fmea c. Implement the fmea into the skystash ecosystem d. validate 9. state the hypotheses ○ Proposal SHM ○ Detect errors and report them 10. outline the order of information in the thesis a. Descriptors b. Fmea c. Skystash, metadata
%Q: How to solve this rather extensive problem of SHM?

%Q: whats the context on this work?


%  interoperability
%In conjunction with the FAIR principles this work also focuses on reusability of the components of this work as well as interoperability with outside algorithms. Allowing the results of this work to be accessible by uploading them to the FAIR skystash data-platform and making condensing errors into tags and using conventions to catalyze digitalization to make the resulting data findable.


%2. provide a review of the literature related to the topic
%a. FMEA/data definitions
%b. FAIR
%c. Current skystash/digital twin
%d. soil


% motivation?
% 1. actually detect data errors
% 2. promptly detect errors. manual diagnosis steps --> reduce downtime
% 3. increase quality of flight test data by marking errors to detect them




