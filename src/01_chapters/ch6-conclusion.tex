\chapter{Conclusion}
Development of an application to detect faults in Flight Test Data using conventional, non-ML methods is executed in this work. Firstly, the problem is segmented into three distinct sub-domains with the first being an availability check of sensor data by checking against the flight configuration and detecting whether sensors are available. The second step consists of checking single timeseries with algorithms that do not include other sensor data.  Thirdly, sensor data is checked against each other to take into account redundant sensor data as well as physically correlating sensors. After introducing the general algorithms to detect faults, some attention is given to the software development side and the overall architecture in order to develop a design that fits into the Digital Twin project and interfaces cleanly with the skystash while also honoring FAIR guiding princples. Within this part including configuration files and metadata management, a method is developed to separate the actually generated DAQ-data from additional, general metadata. This method will then allow to allocate various types of metadata as well as SHM-configuration into a single, standardized JSON format. This standardized shape of metadata can then also be used to configure the data processing step, making the kind of checks and algorithms retraceable. This essentially decouples the single steps of the entire process into four steps:

\begin{enumerate}
        \item Upload of raw data and metadata
        \item Metadata Merge
        \item Data Processing and Fault Detection
        \item Fault Mode Analysis
\end{enumerate}

Using these four steps and having put an emphasis on the general architecture of SHM this work still achieves to catch many previously unknown faults and occurrences of interest while still employing relatively simple algorithms.

This approach and general implementation of novel architecture '' cracks open a piñata filled with opportunity ''.\cite{zacharias_what_2023} Enabling algorithms for FMEA to be quickly deployed and evaluated in contrast to previous work-processes in which data accessibility has been a huge issue. At this point, the barrier of entry for other researchers is greatly lowered, facilitating implementation of novel FMEA algorithm and testing it on this new testbed to generate data quality indices while also being able to have access to test cases and other algorithms to compare and validate to. Since especially research data is highly customized this can be a possible way to deal with complex transformation functions as well as complex sensor/parameter correlations to quicker acquire an overview. Particularly the rapidly growing field of IoT may benefit from such an application that allows integration into a data space such as the skystash to automatically evaluate sensors, infrastructures and FMEAs. Quickly generating value without having to implement novel infrastructure.
Additionally, it lays the groundwork for a metadata-driven system monitoring which has the potential to be expanded to any system generating time-series sensordata.
%-can also be used without cloudmodel and just on straight sensordata

	%\section{Stacked V-Model}
%This work per se can be modeled within a stacked v-diagram like in figure (stacked-v).

%\section{What has been done and accomplished in this work}

%\section{What modular interfaces were built that can be extended in the future}



