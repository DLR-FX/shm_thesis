\chapter{Conclusion}


\section{Discussion/Potential}

-lvl2:Of course, this topic could be examined within a depth that may exceed this work such as estimating white noise using discrete fourier Transform Subspace Decompositions \cite{hendriks_noise_2008} or statistical methods such as covariance operators.

The goal of this work however, is preservation of scalability, minimal user and configuration inputs.

Implementation list:tryouts
-lvl3 rigid body simulation for flight dynamics
-lvl3 implementation of multiple COS and their respective conversions
-lvl3 residual interpretation (better than highpass filtering and checking for std)

-Real time implementation
-Pseudo transfer functions (Generate a real time correlation matrix (MArkov parameters) and generate residuals based on that)
-PCA
-better parameter fusion algorithms (voting, or others see GNSS-RAIM)
        -aaim,
        -external factors integrity monitoring
- GANomaly approach (neural network) since an aircraft dataset is similar to images in that it is a multidimensional dataset

-more and better evaluation of the report display to satisfy innovation turbine requirements.


builds foundation for iot platforms that automatically can evaluate sensors, infrastructures and FMEAs. Since especially research data is highly customized this can be a possible way to deal with complex transformation functions as well as complex sensor/parameter interrelations.

	%\section{Stacked V-Model}
%This work per se can be modeled within a stacked v-diagram like in figure (stacked-v).

%\section{What has been done and accomplished in this work}

%\section{What modular interfaces were built that can be extended in the future}



